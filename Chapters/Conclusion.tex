\chapter{Konklusjon}

Formålet med dette prosjektet var å digitalisere en treningsdagbok for skyttere. Dette ble digitalisert i form av en nettside, som skaleres godt på både PC, Tablets og Smartphones. Det originale målet bak var å kunne lage en nettside som kunne bli sluppet som en beta kandidat etter endt prosjektperiode, hvor videreutviklingen kan fortsette ut ifra tilbakemeldinger fra brukere. I dag er nettsiden fortsatt i en alpha versjon, og har fortsatt noe mangler før en beta periode kan begynne. 


Det finnes mange treningsdagbøker tilgjengelige i dag, men ingen direkte mot skyttere, derfor er dette den første digitale treningsdagboken som kan benyttes av skyttere til å loggføre sine treninger og resultater. Ved å benytte denne treningsdagboken vil skyttere kunne få et større utbytte fra sine økter, samle alle sine trenings og resultats logger på ett sted, og se visuelle grafer over data lagt inn i systemet.


Videre arbeid vil være å gjøre ytterligere kvalitetskontroll over systemet, som innebærer å lage automatiske skript for tester. Videre gjøre eksperimenter mot en eventuell fremtidig Smartphone applikasjon, finne ut av eventuelle forandringer som må skje for å gjøre dette mulig, og gjøre dagboken tilgjengelig for skyttere i form av beta. Deretter burde utviklingen følge tilbakemeldinger fra skyttere.


