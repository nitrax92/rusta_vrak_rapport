\chapter{Løsning}
\lhead{Løsning}

\section{Krav}
\subsection{Ikke-funksjonelle krav}

\section{Designspesifikasjoner}



\section{Implementasjon}






\newpage
\section{Priser og Rabatt}
Rusta Vrak Bilutleige ønsker at når en kunde først skal leie en bil, skal det være så lenge som mulig for å slippe å hente og levere biler mer enn nødvendig. Derfor vil bedriften gi kunder gode rabatter som øker i forhold til lenge på leieperiode. Leieprisen av de aller fleste bilene ligger på 250kr pr. døgn, og i dag tilbyr bedriften utleie for 30 dager for kun 2500kr. Dette tilsvarer en prisreduksjon på 66.6\%. For å kunne implementere et liknende system på nettsiden ble det laget en rekke løsningsforslag i Excel, dette finner man vedlagt i vedlegg OOAWNDO

\subsection*{Valgte Løsning}
Den valgte løsningen går ut på å gi kunden 17.5\% ekstra avslag hver 5. dag i leieperioden. Se tabell \ref{table:percent} for en oversikt over hvordan pris avslaget stiger. Figur \ref{fig:price_reduction} viser hvordan prisantydningen blir på biler som har en døgnpris på 250kr. Dette vil kunne bidra til at en kunde leier litt ekstra kun for å få ekstra avslag på prisen. 

\begin{table}[htbp]
\centering
\caption{Rabatt man oppnår ved leieperioder}
\label{table:percent}
\begin{tabular}{|l|l|l|l|l|l|l|l|l}
\cline{1-8}
Antall Dager & 0-4   & 5-9    & 10-14     & 15-19    & 20-24    & 25-29    & 30      &  \\ \cline{1-8}
Rabatt       & 100\% & 82.5\% & 68.0625\% & 56.151\% & 46.325\% & 38.218\% & 33.33\% &  \\ \cline{1-8}
\end{tabular}
\end{table}



 \begin{figure}[htbp]
	\centering
		\includegraphics[scale=0.5]{Bilder/avslag.png}
	\caption[Utleiepris Diagram]{Oversikt over rabatt som tilføres i forhold til hvor lenge en bil blir utleid. } %\ref{fig:iterative}
	\label{fig:price_reduction}
\end{figure}


\clearpage
\section{Admin}

\clearpage
\section{Database}
Dette prosjektet benytter Django, og dermed vil dette rammeverket ta hånd om mye av arbeidet mot databasen. Django kommer utstyrt med en ORM (Object Relational Mapper), som håndterer overgangen fra Python klasseer til MySQL tabeller. 

\clearpage
\section{Testing og Validering}



